\documentclass[a4j,12pt]{jarticle}
\usepackage[dvipdfm,dvipdfmx,dvips]{graphicx}
\usepackage{amssymb}
\usepackage{url}
%\title{魔方陣全解出力アルゴリズムの改良と \\ 超並列計算環境でのプログラム実行}
\title{魔方陣全解出力アルゴリズムの改良}
\author{
	茨城県立並木中等教育学校 4年 \\ 杉崎 行優
}
\date{}

%From http://d.hatena.ne.jp/Rion778/touch/20091002/1254482262
\makeatother
\def\linesparpage#1{
    \baselineskip=\textheight
    \divide\baselineskip by #1
}

\begin{document}

\linesparpage{40}

\maketitle

\thispagestyle{empty}
%do not count the surface page in 10 pages
\setcounter{page}{0}

%\tableofcontents
%\listoffigures
\newpage

%For JSEC
\def\thesection{\Alph{section}}
\setcounter{section}{0}

%\section{要旨、概要}
%学校のPC室のシステムを利用して並列計算システムを構築した。
%また、そのシステムの上でベンチマークソフトを動かし、システムの性能評価を行った。
%その結果、計算によっては通信によるオーバーヘッドが見られた。
%
%現在は、初期段階のシステムを進化させ、Namiki Linuxと名付けた並列計算システムの開発を行っている。


%\section{問題提起、研究目的}
\section{研究目的}
前回の研究で、
学校のコンピュータ室のコンピュータを用いて並列計算システムの構築をした。
そのシステム上で、以前から開発していた、4次魔方陣の全解を
総当たりで探求する並列プログラムを走らせたところ、
一晩かけても実行が終了しなかった。

その後、筑波大学のスーパーコンピュータT2K-Tsukuba学際開拓プログラムに申し込み、
そのプログラムをT2K-Tsukubaの512CPUコア上で走らせた。
しかし、最大実行可能時間である24時間をかけても実行は終了しなかった。

そのことから、アルゴリズムを改良し魔方陣の全解を
一般のコンピュータやT2K-Tsukuba上で探求しようと思った。

\section{研究方法}
この研究は2012年5月から2013年9月にかけて行った。
今回は魔方陣の全解を探求し出力するプログラムを開発した。
アルゴリズムとしては総当たりを用い、その中で枝刈り法を改良した。

また、並列計算ライブラリにはMPI(Message passing interface)を用いた。
魔方陣の総数はどれほど多くなるか分からなかったので、
オーバーフローしないよう、カウンターには
GMP(GNU multiple precision arithmetic library)で提供されている
{\tt mpz\_t}型の変数を使用した。

\subsection{用語・定数の定義}
$X$を自然数とする。縦$X$、横$X$の計$X^2$マスに整数が入れられているものを
「$X$次の陣」と呼ぶ。
また、陣の内、$1$から$X^2$の数が1つずつ用いられており、
縦の$X$列、横の$X$列、斜めの$2$列それぞれの合計が全て等しくなるものを「$X$次の魔方陣」と呼ぶ。
(ただし$L$は式\ref{eqn:ms-oneline}によって定義される)

\begin{equation} \label{eqn:ms-oneline}
L=\frac{1}{X} \sum_{i=1}^{X^2}i = \frac{1}{2} X(X^2+1)
\end{equation}


\subsection{アルゴリズムの改良}
この方法では以下の手順を行った。
\begin{enumerate}
	\item 効率のよい枝刈り法を考案する。
	\item 仮計算なども行いながら、そのアルゴリズムの大まかな計算量を、正否判定する陣の数などから求める。
	\item そのアルゴリズムをプログラム化し、一般のコンピュータ1台上で実行する。
	\item 実行のプロファイリング等を用いて、冗長的な処理等を分析し、
	 	次のアルゴリズム作成に生かす。
	\item 最も効率のよいプログラムよかったプログラムを並列化し、T2K-Tsukuba上で実行する。
	\item 実行時の処理性能低下などを分析し、原因を分析する。
\end{enumerate}


%\subsection{一般のコンピュータ1台での実行}
%今回計算に用いた一般のコンピュータの性能を表\ref{tb:normalpc-perf}に示す。
%
%\begin{table}[htb]
	%\begin{center}
	%\begin{tabular}{|l|l|}
%\hline \hline
%\multicolumn{1}{|c|}{項目} & \multicolumn{1}{|c|}{仕様} \\
%\hline \hline
%CPU & Intel Pentium 4 \ 3.20GHz \\
%CPUコア数 & 2 cores \\
%CPUピーク性能 & 6.40GFLOPS \\
%CPUメモリ & 3GB \\
%OS & Linux 3.2.46 \\
%コンパイラ & GCC 4.7.2 \\
%\hline
	%\end{tabular}
	%\end{center}
	%\caption{一般のコンピュータの性能概要}
	%\label{tb:normalpc-perf}
%\end{table}
%
%なお、実行には2CPUコアのうち1CPUコアのみを用いた。


\subsection{T2K-Tsukubaの利用}
T2K-Tsukubaとは、筑波大学に置かれているスーパーコンピュータの1つである。
%その性能を表\ref{tb:t2k-perf}に示す。
%
%\begin{table}[htb]
	%\begin{center}
	%\begin{tabular}{|l|l|l|}
%\hline \hline
%\multicolumn{1}{|c|}{区分} & \multicolumn{1}{|c|}{項目} & \multicolumn{1}{|c|}{仕様} \\
%\hline \hline
%計算ノード & CPU & quad-core Opteron 2.3GHz $\times$ 4台 \\
%& CPUコア数 & 16 cores \\
%& CPUメモリ & 32GB \\
%& OS & Linux 2.6.18 \\
%& コンパイラ & GCC 4.1.2 \\
%& MPIソフトウェア & MVAPICH2 1.4.1 \\
%\hline
%並列ネットワーク & ネットワーク & Infiniband 4$\times$DDR \\
%& 1本当たりのピーク性能(片方向) & 2GB/s \\
%& トポロジ & Fat-Tree \\\hline
%全体システム構成 & 総ノード数 & 648台 \\
%& CPUピーク性能 & 95.4TFLOPS \\
%& ファイルシステム & 800TB(Lustre, RAID-6) \\\hline
	%\end{tabular}
	%\end{center}
 	%\caption{T2K-Tsukubaの性能概要}
	%\label{tb:t2k-perf}
%\end{table}

今回は、この研究のために学際開拓プログラムに申し込み、T2K-Tsukubaを利用させていただいた。
一度に利用できる最大ノード数は32台となった。
また、最大連続実行可能時間は24時間となっており、実行時間がこの時間を超えた場合、
そのジョブは強制終了される。

並列方式にはマスタ・ワーカー型並列を用いた。
実行は全$16 \times 32 = 512$プロセスの内、1プロセスをマスタとしその他をワーカーとした。

T2K-Tsukubaでのプログラム実行の手順は以下の様になっている。
\begin{enumerate}
	\item ログインノードにSSH(Secure shell)を用いてログインする。
	\item プログラムを{\tt mpicc}コマンドでコンパイルし、
	実行ファイルを作成する。 \label{enum:howtouse-t2k:compile}
	\item 以下の手順を行うシェルスクリプトを書き、{\tt qsub2}コマンドにて
 	ジョブをサブミットする。
  	\begin{enumerate}
   		\item プログラム実行に使用するノードの台数と最大実行時間
     		を指定する。
   		\item 使用するノードをリストアップするスクリプトを
     		実行する。
       		\item {\tt mpirun\_rsh}コマンドでプログラム
         	(手順\ref{enum:howtouse-t2k:compile}で作成した実行ファイル)
          	を実行する。
   	\end{enumerate}
    	\item その後、ジョブは逐次自動実行される。
\end{enumerate}

%\section{結果}
\section{得られた結果}
\subsection{アルゴリズムの開発・改良}
\subsubsection{第一次アルゴリズム}
このアルゴリズムでは完全総当たりの方法を用いた。
具体的には、1つ1つのマスに$1$から$n^2$の数を順に入れていき、全てのマスを埋めたら
正否を判定するというものだ。
最終的に判定される陣数は$(X^2)^{(X^2)}$である。

このアルゴリズムをプログラム化し、一般のコンピュータで実行したところ、
3次の場合、実行時間が約20秒となった。
このプログラムで4次の場合の計算をすると、式\ref{eqn:prog1-exectime4}より、
実行に約$1.4 \times 10^{12}$秒、つまり約$4.4 \times 10^4$年かかることになる。
よって、4次以上の場合は現実的な時間では実行終了しないので、実際に実行はしなかった。

\begin{equation} \label{eqn:prog1-exectime4}
20 \times ((4^2)^{(4^2)} \div (3^2)^{(3^2)}) \approx 1.4 \times 10^{12}
\end{equation}


\subsubsection{第二次アルゴリズム}
このアルゴリズムでは動的計画法を用いた。
具体的には、あらかじめ合計が$L$である要素$X$の数列(正和列と呼ぶ)を作成しておき、
それらの中から$X$個選び、それぞれを陣の横列として代入するというものだ。
正和列は順列で表される。
また、各$X$での正和列の個数$E$を表\ref{tab:ple-each-X}に示す。
最終的に判定される陣数は$_E \mathrm{P} _X$である。

\begin{table}[htb]
	\begin{center}
	\begin{tabular}{|l|r|}
\hline \hline
\multicolumn{1}{|c|}{$X$} & \multicolumn{1}{|c|}{$E$} \\
\hline \hline
3 & $48$ \\
4 & $2,064$ \\
5 & $167,280$ \\
6 & $23,136,480$ \\
\hline
	\end{tabular}
	\end{center}
	\caption{各$X$における正和列の個数$E$}
	\label{tab:ple-each-X}
\end{table}

このアルゴリズムをプログラム化し、一般のコンピュータで実行したところ、
3次の場合、実行時間が約0.05秒となった。
このプログラムで4次の場合の計算をすると、式\ref{eqn:prog2-exectime4}より
実行に$8.7 \times 10^6$秒、つまり約$101$日かかることになる。
よって、このプログラムも
4次以上の場合は現実的な時間で実行終了しないので、実際に実行はしなかった。

\begin{equation} \label{eqn:prog2-exectime4}
0.05 \times (_{2064} \mathrm{P} _4 \div _{48} \mathrm{P} _3) \approx 8.7 \times 10^6
\end{equation}

\subsubsection{第三次アルゴリズム}
このアルゴリズムでは、第一次アルゴリズムのように陣に数字を埋めていく際、
列中の$X-1$個の数が埋まれば、残りの$1$マスは$L$からそれらの数の合計を引くことにより
求められることを利用した。

さらに今回は、数字を埋める順番を工夫した。
まず初めに斜めのマスを埋め、次にその他のマスを埋めていくというものだ。
その手順を図\ref{pic:prog3-orderofchain}に示す
(丸番号は総当りに埋められるマスの順番、
チェックマークは引き算により求められるマスを表している)。
こうすることにより、より多くのマスを自動的に埋められるようになった。
これは、斜めのマスは他のマスに比べ、斜めの列上のマスを自動的に埋めるというTODO


\subsection{T2K-Tsukuba上での実行}
今回は第三次アルゴリズムのプログラムを並列実行した。
並列方式にはマスタ・ワーカー型並列を用いた。
その手順を以下に示す。
\begin{enumerate}
\item マスタが$N$段目($N \in \mathbb{N}$、ただし$N$は総当たりされるマスの数以下)
まで数字の総当たりをする。
\item そのうちの1つのパターンを1つのワーカに配る。 \label{enum:master-sending}
\item ワーカーがそれを受け取る。 \label{enum:worker-receiving}
\item そのワーカーが$N+1$段目からの数字を総当たりする。
マスタは手順\ref{enum:master-sending}から繰り返す。
\item 1つのパターンに対するワーカーの処理が終了したら結果をファイルに出力し、
マスタからのパターン配布を待ち、手順\ref{enum:worker-receiving}から繰り返す。
\item マスタの$N$段目までの総当たり処理が終了したらマスタが全ワーカーに終了を伝え、
実行を終了する。
\end{enumerate}

\section{考察}
\subsection{第四次アルゴリズムにおける
		判定される陣から導かれる
		予測実行時間の比と実際の実行時間の比の相違}
	ここで表\ref{tab:bohemian-rhapsody}に、
	3次の場合に対する4次の場合、
	4次の場合に対する5次の場合、
	3次の場合に対する5次の場合
	それぞれの予測実行時間と実際の実行時間の比を示す。

\begin{table}[htb]
	\begin{center}
	\begin{tabular}{|l||r|r|}
\hline
& \multicolumn{1}{|c|}{予測実行時間の比} & \multicolumn{1}{|c|}{実際の実行時間の比} \\
\hline\hline
3次の場合に対する4次の場合 & $1.0 \times 10^6$  & $2.4 \times 10^0$ \\\hline
4次の場合に対する5次の場合 & $7.5 \times 10^8$  & $6.5 \times 10^6$ \\\hline
3次の場合に対する5次の場合 & $7.7 \times 10^{14}$  & $1.6 \times 10^7$ \\\hline
	\end{tabular}
	\end{center}
	\caption{理想実行時間の比と実際の実行時間の比}
	\label{tab:bohemian-rhapsody}
\end{table}

表より、それぞれの予測実行時間の比と
実際の実行時間の比は異なっていることが分かる。
これは、3次と4次の場合の実行時間がとても短く、
時間計測で誤差が生じてしまったことが原因だと考えられる。


\subsection{第四次アルゴリズムのプログラムをT2K-Tsukuba上で
		実行した際の、$N$の値による実行速度の違い}
$N=3$では、各ノードの全体実行時間にばらつきが出た。
これは、$N=3$では問題が細かく分けられていなく、
マスタから受け取る陣によってワーカーの処理時間が異なることにより、
結果として各ワーカーの全体実行時間がつり合うように
仕事を分散できていないからだと考えられる。

また、$N=8$では、各ワーカーの総通信時間が50分ほどであった。
これは、$N=8$では問題が細かく分けられすぎていて、
マスタがワーカーに陣を配る回数が多くなり、
結果として通信をする際の
固定的処理時間が長くなってしまったからだと考えられる。

上記を考えると、
$N=6$の場合が、最も問題を分ける細かさと通信回数のバランスが取れており、
結果として最も実行時間が速くなったのだと考えられる。

%\section{結論(課題)}
\section{結論}
\begin{itemize}
\item 明らかに成立しない陣の探索を減らすようにアルゴリズムやプログラムを改良することにより、
	実行時間を短縮することができる。
\item アルゴリズムやプログラムの枝刈り法を改良することによって実行時間を短縮することができた。
\end{itemize}


%Because it is for JSEC
\section{謝辞}
\begin{itemize}
	\item 筑波大学計算科学研究センター 朴 泰祐 教授	\\
%		T2K-Tsukubaの利用申請やプログラムなどについて多くのご教示をいただきました。 
		アルゴリズムの内容などについて多くのご教示をいただきました。 

	\item 産業技術総合研究所 山本 直孝 様	\\
		コンピュータに関わるお仕事をされている方として、細かい質問などをさせていただきました。

	\item 茨城県立並木中等教育学校 齋藤 達也 先生	\\
		研究・実験の方向性の指導をしてくださいました。

	\item 茨城県立並木中等教育学校 粉川 雄一郎 先生	\\
		アルゴリズムの開発や研究のまとめにおいて助言をいただきました。
\end{itemize}
ここに感謝いたします。

\section{参考文献}
\begin{itemize}
	\item ``計算設備紹介'' \url{http://www.ccs.tsukuba.ac.jp/CCS/research/computer}	\\
		著者: 計算科学研究センター	\\
		閲覧日: 2013年9月24日
\end{itemize}



%なし


%\input{9.pic}
\end{document}
