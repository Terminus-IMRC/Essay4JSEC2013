\section{研究方法}
この研究は2012年5月から2013年9月にかけて行った。
今回は魔方陣の全解を探求し出力するプログラムを開発した。
アルゴリズムとしては総当たりを用い、その中で枝刈り法を改良した。

また、並列計算ライブラリにはMPI(Message passing interface)を用いた。
魔方陣の総数はどれほど多くなるか分からなかったので、
オーバーフローしないよう、カウンターには
GMP(GNU multiple precision arithmetic library)で提供されている
{\tt mpz\_t}型の変数を使用した。

\subsection{用語・定数の定義}
$X$を自然数とする。縦$X$、横$X$の計$X^2$マスに整数が入れられているものを
「$X$次の陣」と呼ぶ。
また、陣の内、$1$から$X^2$の数が1つずつ用いられており、
縦の$X$列、横の$X$列、斜めの$2$列それぞれの合計が全て等しくなるものを「$X$次の魔方陣」と呼ぶ。
(ただし$L$は式\ref{eqn:ms-oneline}によって定義される)

\begin{equation} \label{eqn:ms-oneline}
L=\frac{1}{X} \sum_{i=1}^{X^2}i = \frac{1}{2} X(X^2+1)
\end{equation}


\subsection{アルゴリズムの改良}
この方法では以下の手順を行った。
\begin{enumerate}
	\item 効率のよい枝刈り法を考案する。
	\item 仮計算なども行いながら、そのアルゴリズムの大まかな計算量を、正否判定する陣の数などから求める。
	\item そのアルゴリズムをプログラム化し、一般のコンピュータ1台上で実行する。
	\item 実行のプロファイリング等を用いて、冗長的な処理等を分析し、
	 	次のアルゴリズム作成に生かす。
	\item 最も効率のよいプログラムよかったプログラムを並列化し、T2K-Tsukuba上で実行する。
	\item 実行時の処理性能低下などを分析し、原因を分析する。
\end{enumerate}


%\subsection{一般のコンピュータ1台での実行}
%今回計算に用いた一般のコンピュータの性能を表\ref{tb:normalpc-perf}に示す。
%
%\begin{table}[htb]
	%\begin{center}
	%\begin{tabular}{|l|l|}
%\hline \hline
%\multicolumn{1}{|c|}{項目} & \multicolumn{1}{|c|}{仕様} \\
%\hline \hline
%CPU & Intel Pentium 4 \ 3.20GHz \\
%CPUコア数 & 2 cores \\
%CPUピーク性能 & 6.40GFLOPS \\
%CPUメモリ & 3GB \\
%OS & Linux 3.2.46 \\
%コンパイラ & GCC 4.7.2 \\
%\hline
	%\end{tabular}
	%\end{center}
	%\caption{一般のコンピュータの性能概要}
	%\label{tb:normalpc-perf}
%\end{table}
%
%なお、実行には2CPUコアのうち1CPUコアのみを用いた。


\subsection{T2K-Tsukubaの利用}
T2K-Tsukubaとは、筑波大学に置かれているスーパーコンピュータの1つである。
%その性能を表\ref{tb:t2k-perf}に示す。
%
%\begin{table}[htb]
	%\begin{center}
	%\begin{tabular}{|l|l|l|}
%\hline \hline
%\multicolumn{1}{|c|}{区分} & \multicolumn{1}{|c|}{項目} & \multicolumn{1}{|c|}{仕様} \\
%\hline \hline
%計算ノード & CPU & quad-core Opteron 2.3GHz $\times$ 4台 \\
%& CPUコア数 & 16 cores \\
%& CPUメモリ & 32GB \\
%& OS & Linux 2.6.18 \\
%& コンパイラ & GCC 4.1.2 \\
%& MPIソフトウェア & MVAPICH2 1.4.1 \\
%\hline
%並列ネットワーク & ネットワーク & Infiniband 4$\times$DDR \\
%& 1本当たりのピーク性能(片方向) & 2GB/s \\
%& トポロジ & Fat-Tree \\\hline
%全体システム構成 & 総ノード数 & 648台 \\
%& CPUピーク性能 & 95.4TFLOPS \\
%& ファイルシステム & 800TB(Lustre, RAID-6) \\\hline
	%\end{tabular}
	%\end{center}
 	%\caption{T2K-Tsukubaの性能概要}
	%\label{tb:t2k-perf}
%\end{table}

今回は、この研究のために学際開拓プログラムに申し込み、T2K-Tsukubaを利用させていただいた。
一度に利用できる最大ノード数は32台となった。
また、最大連続実行可能時間は24時間となっており、実行時間がこの時間を超えた場合、
そのジョブは強制終了される。

並列方式にはマスタ・ワーカー型並列を用いた。
実行は全$16 \times 32 = 512$プロセスの内、1プロセスをマスタとしその他をワーカーとした。

T2K-Tsukubaでのプログラム実行の手順は以下の様になっている。
\begin{enumerate}
	\item ログインノードにSSH(Secure shell)を用いてログインする。
	\item プログラムを{\tt mpicc}コマンドでコンパイルし、
	実行ファイルを作成する。 \label{enum:howtouse-t2k:compile}
	\item 以下の手順を行うシェルスクリプトを書き、{\tt qsub2}コマンドにて
 	ジョブをサブミットする。
  	\begin{enumerate}
   		\item プログラム実行に使用するノードの台数と最大実行時間
     		を指定する。
   		\item 使用するノードをリストアップするスクリプトを
     		実行する。
       		\item {\tt mpirun\_rsh}コマンドでプログラム
         	(手順\ref{enum:howtouse-t2k:compile}で作成した実行ファイル)
          	を実行する。
   	\end{enumerate}
    	\item その後、ジョブは逐次自動実行される。
\end{enumerate}
