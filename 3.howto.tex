\section{研究方法}
この研究は2012年5月から2013年9月にかけて行った。
今回は魔方陣の全解を探求し出力するプログラムを開発した。
アルゴリズムに関しては、総当たりより効率のよい計算手法ものは
発見されていないので、総当たりの手法を用いた。

また、並列計算ライブラリにはMPI(Message passing interface)を用いた。
魔方陣の総数はどれほど多くなるか分からなかったので、
オーバーフローしないよう、カウンターには
GMP(GNU multiple precision arithmetic library)で提供されている
{\tt mpz\_t}型の変数を使用した。

\subsection{用語・定数の定義}
$X$を自然数とする。縦$X$、横$X$の計$X^2$マスに整数が入れられているものを
「$X$次の陣」と呼ぶ。
また、陣の内、$1$から$X^2$の数が1つずつ用いられており、
縦の$X$列、横の$X$列、斜めの$2$列それぞれの合計が全て等しく
式\ref{eqn:ms-oneline}($L$と定義する)となるものを「$X$次の魔方陣」と呼ぶ。

\begin{equation} \label{eqn:ms-oneline}
L=\sum_{i=1}^{X^2}i \div X = X(X^2+1) \div 2
\end{equation}


\subsection{アルゴリズムの改良}
この方法では以下の手順を行った。
\begin{enumerate}
	\item 新しいアルゴリズムを考案する。
	\item 仮計算なども行いながら、そのアルゴリズムの大まかな計算量を、正否判定する陣の数などから求める。
	\item そのアルゴリズムをプログラム化し、一般のコンピュータやT2K-Tsukuba上で実行する。
	\item プロファイリング等を用いて、オーバーヘッドとなっている処理等を分析し、
 	次のアルゴリズム作成に生かす。
\end{enumerate}

\subsection{一般のコンピュータでの実行}
今回計算に用いた一般のコンピュータの性能を表\ref{tb:normalpc-perf}に示す。

\begin{table}[htb]
	\begin{center}
	\begin{tabular}{|l|l|}
\hline \hline
\multicolumn{1}{|c|}{項目} & \multicolumn{1}{|c|}{仕様} \\
\hline \hline
CPU & Intel Pentium 4 3.20GHz \\
CPUコア数 & 2 cores \\
CPUピーク性能 & 6.40GFLOPS \\
CPUメモリ & 3GB \\
OS & Linux 3.2.46 \\
コンパイラ & GCC 4.7.2 \\
\hline
	\end{tabular}
	\end{center}
	\caption{一般のコンピュータの性能概要}
	\label{tb:normalpc-perf}
\end{table}

なお、実行にはCPUの方コアのみを用いた。


\subsection{T2K-Tsukubaの利用}
T2K-Tsukubaとは、筑波大学に置かれているスーパーコンピュータの1つである。
その性能を表\ref{tb:t2k-perf}に示す。

\begin{table}[htb]
	\begin{center}
	\begin{tabular}{|l|l|l|}
\hline \hline
\multicolumn{1}{|c|}{区分} & \multicolumn{1}{|c|}{項目} & \multicolumn{1}{|c|}{仕様} \\
\hline \hline
計算ノード & CPU & quad-core Opteron 2.3GHz $\times$ 4台 \\
& CPUコア数 & 16 cores \\
& CPUメモリ & 32GB \\
& OS & Linux 2.6.18 \\
& コンパイラ & GCC 4.1.2 \\
& MPIソフトウェア & MVAPICH2 1.4.1 \\
\hline
並列ネットワーク & ネットワーク & Infiniband 4$\times$DDR \\
& 1本当たりのピーク性能(片方向) & 2GB/s \\
& トポロジ & Fat-Tree \\\hline
全体システム構成 & 総ノード数 & 648台 \\
& CPUピーク性能 & 95.4TFLOPS \\
& ファイルシステム & 800TB(Lustre, RAID-6) \\\hline
	\end{tabular}
	\end{center}
 	\caption{T2K-Tsukubaの性能概要}
	\label{tb:t2k-perf}
\end{table}

今回は、この研究のために学際開拓プログラムに申し込み、T2K-Tsukubaを利用させていただいた。
一度に利用できる最大ノード数は32台となった。
また、最大連続実行可能時間は24時間となっており、実行時間がこの時間を超えた場合、
そのジョブは強制終了される。

並列方式にはマスタ・ワーカー型並列を用いた。
実行は全$16 \times 32 = 512$プロセスの内、1プロセスをマスタとしその他をワーカーとした。

T2K-Tsukubaでのプログラム実行の手順は以下の様になっている。
\begin{enumerate}
	\item ログインノードにSSH(Secure shell)を用いてログインする。
	\item プログラムを{\tt mpicc}コマンドでコンパイルし、
	実行ファイルを作成する。 \label{enum:howtouse-t2k:compile}
	\item 以下の手順を行うシェルスクリプトを書き、{\tt qsub2}コマンドにて
 	ジョブをサブミットする。
  	\begin{enumerate}
   		\item プログラム実行に使用するノードの台数と最大実行時間
     		を指定する。
   		\item 使用するノードをリストアップするスクリプトを
     		実行する。
       		\item {\tt mpirun\_rsh}コマンドでプログラム
         	(手順\ref{enum:howtouse-t2k:compile}で作成した実行ファイル)
          	を実行する。
   	\end{enumerate}
    	\item その後、ジョブは逐次自動実行される。
\end{enumerate}
