\section{考察}
\subsection{第四次アルゴリズムにおける
		判定される陣から導かれる
		予測実行時間の比と実際の実行時間の比の相違}
	ここで表\ref{tab:bohemian-rhapsody}に、
	3次の場合に対する4次の場合、
	4次の場合に対する5次の場合、
	3次の場合に対する5次の場合
	それぞれの予測実行時間と実際の実行時間の比を示す。

\begin{table}[htb]
	\begin{center}
	\begin{tabular}{|l||r|r|}
\hline
& \multicolumn{1}{|c|}{予測実行時間の比} & \multicolumn{1}{|c|}{実際の実行時間の比} \\
\hline\hline
3次の場合に対する4次の場合 & $1.0 \times 10^6$  & $2.4 \times 10^0$ \\\hline
4次の場合に対する5次の場合 & $7.5 \times 10^8$  & $6.5 \times 10^6$ \\\hline
3次の場合に対する5次の場合 & $7.7 \times 10^{14}$  & $1.6 \times 10^7$ \\\hline
	\end{tabular}
	\end{center}
	\caption{理想実行時間の比と実際の実行時間の比}
	\label{tab:bohemian-rhapsody}
\end{table}

表より、それぞれの予測実行時間の比と
実際の実行時間の比は異なっていることが分かる。
これは、3次と4次の場合の実行時間がとても短く、
時間計測で誤差が生じてしまったことが原因だと考えられる。


\subsection{第四次アルゴリズムのプログラムをT2K-Tsukuba上で
		実行した際の、$N$の値による実行速度の違い}
$N=3$では、各ノードの全体実行時間にばらつきが出た。
これは、$N=3$では問題が細かく分けられていなく、
マスタから受け取る陣によってワーカーの処理時間が異なることにより、
結果として各ワーカーの全体実行時間がつり合うように
仕事を分散できていないからだと考えられる。

また、$N=8$では、各ワーカーの総通信時間が50分ほどであった。
これは、$N=8$では問題が細かく分けられすぎていて、
マスタがワーカーに陣を配る回数が多くなり、
結果として通信をする際の
固定的処理時間が長くなってしまったからだと考えられる。

上記を考えると、
$N=6$の場合が、最も問題を分ける細かさと通信回数のバランスが取れており、
結果として最も実行時間が速くなったのだと考えられる。
