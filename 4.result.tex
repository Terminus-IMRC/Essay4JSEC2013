%\section{結果}
\section{得られた結果}
\subsection{アルゴリズムの開発・改良}
\subsubsection{第一次アルゴリズム}
このアルゴリズムでは完全総当たりの方法を用いた。
具体的には、1つ1つのマスに$1$から$n^2$の数を順に入れていき、全てのマスを埋めたら
正否を判定するというものだ。
最終的に判定される陣数は$(X^2)^{(X^2)}$である。

このアルゴリズムをプログラム化し、一般のコンピュータで実行したところ、
3次の場合、実行時間が約20秒となった。
このプログラムで4次の場合の計算をすると、式\ref{eqn:prog1-exectime4}より、
実行に約$1.4 \times 10^{12}$秒、つまり約$4.4 \times 10^4$年かかることになる。
よって、4次以上の場合は現実的な時間では実行終了しないので、実際に実行はしなかった。

\begin{equation} \label{eqn:prog1-exectime4}
20 \times ((4^2)^{(4^2)} \div (3^2)^{(3^2)}) \approx 1.4 \times 10^{12}
\end{equation}


\subsubsection{第二次アルゴリズム}
このアルゴリズムでは動的計画法を用いた。
具体的には、あらかじめ合計が$L$である要素$X$の数列(正和列と呼ぶ)を作成しておき、
それらの中から$X$個選び、それぞれを陣の横列として代入するというものだ。
正和列は順列で表される。
また、各$X$での正和列の個数$E$を表\ref{tab:ple-each-X}に示す。
最終的に判定される陣数は$_E \mathrm{P} _X$である。

\begin{table}[htb]
	\begin{center}
	\begin{tabular}{|l|r|}
\hline \hline
\multicolumn{1}{|c|}{$X$} & \multicolumn{1}{|c|}{$E$} \\
\hline \hline
3 & $48$ \\
4 & $2,064$ \\
5 & $167,280$ \\
6 & $23,136,480$ \\
\hline
	\end{tabular}
	\end{center}
	\caption{各$X$における正和列の個数$E$}
	\label{tab:ple-each-X}
\end{table}

このアルゴリズムをプログラム化し、一般のコンピュータで実行したところ、
3次の場合、実行時間が約0.05秒となった。
このプログラムで4次の場合の計算をすると、式\ref{eqn:prog2-exectime4}より
実行に$8.7 \times 10^6$秒、つまり約$101$日かかることになる。
よって、このプログラムも
4次以上の場合は現実的な時間で実行終了しないので、実際に実行はしなかった。

\begin{equation} \label{eqn:prog2-exectime4}
0.05 \times (_{2064} \mathrm{P} _4 \div _{48} \mathrm{P} _3) \approx 8.7 \times 10^6
\end{equation}

\subsubsection{第三次アルゴリズム}
このアルゴリズムでは、第一次アルゴリズムのように陣に数字を埋めていく際、
列中の$X-1$個の数が埋まれば、残りの$1$マスは$L$からそれらの数の合計を引くことにより
求められることを利用した。

さらに今回は、数字を埋める順番を工夫した。
まず初めに斜めのマスを埋め、次にその他のマスを埋めていくというものだ。
その手順を図\ref{pic:prog3-orderofchain}に示す
(丸番号は総当りに埋められるマスの順番、
チェックマークは引き算により求められるマスを表している)。
こうすることにより、より多くのマスを自動的に埋められるようになった。
これは、斜めのマスは他のマスに比べ、斜めの列上のマスを自動的に埋めるというTODO


\subsection{T2K-Tsukuba上での実行}
今回は第三次アルゴリズムのプログラムを並列実行した。
並列方式にはマスタ・ワーカー型並列を用いた。
その手順を以下に示す。
\begin{enumerate}
\item マスタが$N$段目($N \in \mathbb{N}$、ただし$N$は総当たりされるマスの数以下)
まで数字の総当たりをする。
\item そのうちの1つのパターンを1つのワーカに配る。 \label{enum:master-sending}
\item ワーカーがそれを受け取る。 \label{enum:worker-receiving}
\item そのワーカーが$N+1$段目からの数字を総当たりする。
マスタは手順\ref{enum:master-sending}から繰り返す。
\item 1つのパターンに対するワーカーの処理が終了したら結果をファイルに出力し、
マスタからのパターン配布を待ち、手順\ref{enum:worker-receiving}から繰り返す。
\item マスタの$N$段目までの総当たり処理が終了したらマスタが全ワーカーに終了を伝え、
実行を終了する。
\end{enumerate}
