%\section{問題提起、研究目的}
\section{研究目的}
これまでに
学校のコンピュータ室のコンピュータを用いて並列計算システムの構築をした。
そのシステム上で、4次魔方陣の全解を
総当たりで探求する並列プログラムを実行したところ、
一晩かけても実行が終了しなかった。

その後、筑波大学のスーパーコンピュータT2K-Tsukuba学際利用プログラムに申し込み、
そのプログラムをT2K-Tsukubaの512CPUコア上で走らせた。
しかし、最大実行可能時間である24時間をかけても実行は終了しなかった。

そのことから、アルゴリズムを改良し魔方陣の全解を
一般のコンピュータ1台やT2K-Tsukuba上で探求することを目的とした。
そのために、枝刈り法の改良やより効率のいいアルゴリズムの開発をした。
